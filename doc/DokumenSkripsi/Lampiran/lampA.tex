%versi 2 (8-10-2016)
\chapter{Kode Program}
\label{lamp:A}

%selalu gunakan single spacing untuk source code !!!!!
\singlespacing 
% language: bahasa dari kode program
% terdapat beberapa pilihan : Java, C, C++, PHP, Matlab, R, dll
%
% basicstyle : ukuran font untuk kode program
% terdapat beberapa pilihan : tiny, scriptsize, footnotesize, dll
%
% caption : nama yang akan ditampilkan di dokumen akhir, lihat contoh
\begin{lstlisting}[language=Java,basicstyle=\tiny,caption=ScheduleClass.java,label=lst:ScheduleClass]

import java.time.LocalDate;
import java.time.LocalTime;
import javafx.beans.property.SimpleStringProperty;
import javafx.beans.property.StringProperty;

/**
 *
 * @author Ariq
 */
public class ScheduleClass {
    private LocalDate date;
    private LocalTime timeAwal;
    private LocalTime timeAkhir;
    private StringProperty subject;
    private StringProperty dosen;
    private StringProperty location;
    
    public ScheduleClass (LocalDate date, LocalTime timeAwal
            ,LocalTime timeAkhir, String subject, String dosen
            , String location)
    {
        this.date = date;
        this.timeAwal = timeAwal;
        this.timeAkhir = timeAkhir;
        this.subject = new SimpleStringProperty(subject);
        this.dosen = new SimpleStringProperty(dosen);
        this.location = new SimpleStringProperty(location);
    }

    /**
     * @return the date
     */
    public LocalDate getDate() {
        return date;
    }

    /**
     * @param date the date to set
     */
    public void setDate(LocalDate date) {
        this.date = date;
    }

    /**
     * @return the time
     */
    public LocalTime getTimeAwal() {
        return timeAwal;
    }

    /**
     * @param time the time to set
     */
    public void setTimeAwal(LocalTime timeAwal) {
        this.timeAwal = timeAwal;
    }
    
     /**
     * @return the time
     */
    public LocalTime getTimeAkhir() {
        return timeAkhir;
    }

    /**
     * @param time the time to set
     */
    public void setTimeAkhir(LocalTime timeAkhir) {
        this.timeAkhir = timeAkhir;
    }

    /**
     * @return the subject
     */
    public String getSubject() {
        return subject.get();
    }

    /**
     * @param subject the subject to set
     */
    public void setSubject(String subject) {
        this.subject.set(subject);
    }

    /**
     * @return the dosen
     */
    public String getDosen() {
        return dosen.get();
    }

    /**
     * @param dosen the dosen to set
     */
    public void setDosen(String dosen) {
        this.dosen.set(dosen);
    }

    /**
     * @return the location
     */
    public String getLocation() {
        return location.get();
    }

    /**
     * @param location the location to set
     */
    public void setLocation(String location) {
        this.location.set(location);
    }
    
    public StringProperty subjectProperty()
    {
       return subject;
    }
    
    public StringProperty dosenProperty()
    {
        return dosen;
    }
    public StringProperty location()
    {
        return location;
    }
    
}

\end{lstlisting}

\begin{lstlisting}[language=Java,basicstyle=\tiny,caption=ExcelConverter.java,label=lst:ExcelConverter]
package icalendarconverter;

import java.io.File;
import java.io.FileInputStream;
import java.io.FileNotFoundException;
import java.io.IOException;
import java.text.SimpleDateFormat;
import java.time.LocalDate;
import java.time.LocalTime;
import java.time.format.DateTimeFormatter;
import java.time.format.FormatStyle;
import java.util.ArrayList;
import java.util.Date;
import java.util.Iterator;
import java.util.List;
import java.util.Locale;
import org.apache.poi.ss.usermodel.Cell;
import org.apache.poi.ss.usermodel.FormulaEvaluator;
import org.apache.poi.ss.usermodel.Row;
import org.apache.poi.ss.util.CellRangeAddress;
import org.apache.poi.ss.util.CellReference;
import org.apache.poi.xssf.usermodel.XSSFRow;
import org.apache.poi.xssf.usermodel.XSSFSheet;
import org.apache.poi.xssf.usermodel.XSSFWorkbook;

/**
 *
 * @author Ariq
 */
public class ExcelConverter {
    
   private File pathFile;
    static XSSFRow row;
    private int rowNoIdx;
    private int colMatkulIdx;
    private LocalDate lc;
    private SimpleDateFormat sp;
    private Date date;
    private DateTimeFormatter indoFormatter;
    private LocalTime lt;
    private String subject;
    
    public ExcelConverter(File pathFile)
    {
        this.pathFile = pathFile;
        this.rowNoIdx = 0;
    }

    public List<ScheduleClass> Converter() throws FileNotFoundException, IOException
    {
        ArrayList<ScheduleClass> scheduleList = new ArrayList<>();
        
        FileInputStream fis = new FileInputStream(pathFile);
        
        XSSFWorkbook wb = new XSSFWorkbook(fis);
        XSSFSheet sheet = wb.getSheetAt(0);
        Iterator < Row > rowIterator = sheet.iterator();   
     
        CellRangeAddress add;
        int colNoIdx = 0;
        ArrayList<String> dosen = new ArrayList<>();
        ArrayList<Integer> idxDosen = new ArrayList<>();
        ArrayList<Integer> colDosen = new ArrayList<>();
        ArrayList<String> location = new ArrayList<>();
        int idxNumber = 0;
        ArrayList<Integer> locationIdx = new ArrayList<>();
        outerloop :
        for (int j = 0; j < sheet.getLastRowNum(); j++) {
            row = sheet.getRow(j);
            for (int f = 0; f < row.getLastCellNum(); f++) {
                Cell cell = row.getCell(f);
                if (cell.getCellType() == Cell.CELL_TYPE_STRING && cell.getStringCellValue().equalsIgnoreCase("No.")) {
                    rowNoIdx = j;
                    colNoIdx = cell.getColumnIndex();
                }
                else if (cell.getCellType() == Cell.CELL_TYPE_STRING && cell.getStringCellValue().equalsIgnoreCase("Nama Mata Kuliah"))
                {
                    colMatkulIdx = cell.getColumnIndex();
                    break outerloop;
                }
                     
            }
        }
        outerloop2 :
        for (int i = 0; i < sheet.getLastRowNum(); i++) {
           outerloop :
            for (int j = 0; j < row.getLastCellNum(); j++) {
                row = sheet.getRow(i);
                if (row == null)
                {
                    i = sheet.getLastRowNum();
                    break outerloop2;
                }
                Cell cell = row.getCell(j);
                FormulaEvaluator evaluator = wb.getCreationHelper().createFormulaEvaluator();
                if (cell.getColumnIndex() == colNoIdx && i > rowNoIdx + 3 && cell.getCellType() != Cell.CELL_TYPE_BLANK
                        && evaluator.evaluate(cell).getCellType() != Cell.CELL_TYPE_NUMERIC) {
                    i = sheet.getLastRowNum();
                    break outerloop2;
                }
                if (cell.getColumnIndex() == colNoIdx && i > rowNoIdx+3 
                        && cell.getCellType() == Cell.CELL_TYPE_BLANK )
                {
                    i = i + 1;
                    break outerloop;             
                }
       
                  if (cell.getRowIndex() > rowNoIdx+1 && cell.getColumnIndex() == (colNoIdx + 1)) {
                    if (cell.getCellType() == Cell.CELL_TYPE_BLANK)  
                    {
                        i = i+1;
                        break outerloop;
                    }
                    String delims = "[,. ]";
                    String[] sumary = cell.getStringCellValue().split(delims);
                    for (int l = 0; l < sumary.length; l++) {
                        if (sumary[l].equalsIgnoreCase("Mrt")) {
                            sumary[l] = "3";
                        }
                        if (sumary[l].equalsIgnoreCase("Okt")) {
                            sumary[l] = "10";
                        }
                        if (sumary[l].equalsIgnoreCase("`16")) {
                            sumary[l] = "2016";
                        } 
                    }

                    lc = LocalDate.of(Integer.parseInt(sumary[5]), Integer.parseInt(sumary[3]), Integer.parseInt(sumary[2]));
                }
                if (cell.getRowIndex() > rowNoIdx+1 && cell.getColumnIndex() == (colNoIdx + 2)) {
                    if (cell.getStringCellValue().equalsIgnoreCase("LIBUR"))
                    {   
                        i = i+1;
                        break outerloop;
                    }
                    else
                    {
                        if(cell.getStringCellValue().equalsIgnoreCase("Shift 1") 
                                || cell.getStringCellValue().equalsIgnoreCase("Shift 2"))
                        {
                            CellReference cr = new CellReference(cell.getRowIndex()+1, cell.getColumnIndex());
                            Row row2 = sheet.getRow(cr.getRow());
                            Cell c = row2.getCell(cr.getCol());
                            String delimsJam = "[-]";
                            String[] arrJam = c.getStringCellValue().split(delimsJam);
                            for (int k = 0; k < arrJam.length; k++) {
                                arrJam[k] = arrJam[k].replace('.', ':');
                            }
                            lt = LocalTime.parse(arrJam[0]);                            
                        }
                        else
                        {
                            String delimsJam = "[-]";
                            String[] arrJam = cell.getStringCellValue().split(delimsJam);
                            for (int k = 0; k < arrJam.length; k++) {
                                arrJam[k] = arrJam[k].replace('.', ':');
                            }
                            lt = LocalTime.parse(arrJam[0]);
                        }
                        
                    }
                                        
                }
                if (cell.getRowIndex() > rowNoIdx+1 && cell.getColumnIndex() == colMatkulIdx ) {
                    subject = cell.getStringCellValue();
                }
   
                if (cell.getRowIndex() > rowNoIdx && cell.getColumnIndex() >= colMatkulIdx+1 
                        && cell.getColumnIndex() < row.getLastCellNum()) {
                    if (cell.getCellType() == Cell.CELL_TYPE_NUMERIC) {

                    }
                    if (cell.getCellType() == Cell.CELL_TYPE_STRING) {
                        if (cell.getStringCellValue().contains(":") ) {
                            String[] splt = cell.getStringCellValue().split(":");
                            String[] splt2 = splt[1].split(",");
                            for (int l = 0; l < splt2.length; l++) {
                                dosen.add(splt2[l].trim());
                                location.add("Lab");
                            }
                        } else  {
                            CellReference cr = new CellReference(1, cell.getColumnIndex());
                            Row row2 = sheet.getRow(cr.getRow());
                            Cell c = row2.getCell(cr.getCol());
                            if (!cell.getStringCellValue().isEmpty())
                            {
                                dosen.add(cell.getStringCellValue().trim());
                                location.add(String.valueOf((int) c.getNumericCellValue()).trim());
                            }
                        }

                    }
                    if (cell.getCellType() == Cell.CELL_TYPE_BLANK && cell.getRowIndex() > 2) {
                        CellReference cr = new CellReference(cell.getRowIndex() - 1, cell.getColumnIndex());
                        Row row2 = sheet.getRow(cr.getRow());
                        Cell c = row2.getCell(cr.getCol());
                        CellReference cr2 = new CellReference(1, cell.getColumnIndex());
                        Row row3 = sheet.getRow(cr2.getRow());
                        Cell c2 = row3.getCell(cr2.getCol());
                        if (c.getStringCellValue().contains(":")) {
                            String[] splt = c.getStringCellValue().split(":");
                            String[] splt2 = splt[1].split(",");
                            for (int l = 0; l < splt2.length; l++) {
                                dosen.add("".trim());
                                location.add("");
                            }
                        } else {
                            if (!c.getStringCellValue().isEmpty())
                            {
                               dosen.add("");
                               location.add("");
                            }
                           
                        }
                    }
                }       
            }

            for (int j = 0; j < dosen.size(); j++) {
                scheduleList.add(new ScheduleClass(lc, lt, lt.plusHours(2), subject, dosen.get(j), location.get(j)));
            }
            dosen.clear();
            location.clear();
           
        }
        
        return Mergering(scheduleList);
    }
    
    public List<ScheduleClass> Mergering (ArrayList<ScheduleClass> scheduleList)
    {
        int count = 0;
        ArrayList<ScheduleClass> scheduleListSmt = new ArrayList<>();
        
        for (int i = 0; i < scheduleList.size(); i++) {

           
           if (scheduleList.get(i).getDosen().isEmpty() )
           {
               scheduleListSmt.add(scheduleList.get(i));
           }         
        }
        for (int i = 0; i < scheduleListSmt.size() ; i++) {
              for (int j = 0; j < scheduleList.size(); j++) {
                  if(scheduleList.get(j).equals(scheduleListSmt.get(i)))
                  {
                      scheduleList.remove(j);
                  }
              }
        }
        for (int i = 0; i < scheduleList.size(); i++) {
            outerloop :
            for (int j = 0; j < scheduleListSmt.size(); j++) {
                if (scheduleList.get(i).getDate().equals(scheduleListSmt.get(j).getDate())
                        && scheduleList.get(i).getTimeAwal().equals(scheduleListSmt.get(j).getTimeAwal()))
                {
                    String ss = scheduleList.get(i).getSubject();
                    scheduleList.get(i).setSubject(ss+", "+scheduleListSmt.get(j).getSubject());
                    j = j + 1;
                    break outerloop;
                }
            }
        }
        return scheduleList;
    }
}

\end{lstlisting}

\begin{lstlisting}[language=Java,basicstyle=\tiny,caption=CalendarConverter.java,label=lst:CalendarConverter]
package icalendarconverter;

import java.io.File;
import java.io.FileNotFoundException;
import java.io.FileOutputStream;
import java.io.IOException;
import java.net.SocketException;
import java.util.GregorianCalendar;
import java.util.List;
import net.fortuna.ical4j.data.CalendarOutputter;
import net.fortuna.ical4j.model.DateTime;
import net.fortuna.ical4j.model.Property;
import net.fortuna.ical4j.model.TimeZone;
import net.fortuna.ical4j.model.TimeZoneRegistry;
import net.fortuna.ical4j.model.TimeZoneRegistryFactory;
import net.fortuna.ical4j.model.ValidationException;
import net.fortuna.ical4j.model.component.VEvent;
import net.fortuna.ical4j.model.component.VTimeZone;
import net.fortuna.ical4j.model.property.CalScale;
import net.fortuna.ical4j.model.property.Description;
import net.fortuna.ical4j.model.property.Location;
import net.fortuna.ical4j.model.property.ProdId;
import net.fortuna.ical4j.model.property.Uid;
import net.fortuna.ical4j.model.property.Version;
import net.fortuna.ical4j.util.UidGenerator;

/**
 *
 * @author Ariq
 */
public class CalendarConverter {
    
    public CalendarConverter()
    {
        
    }
    
    public void calConverter (String path,  ScheduleClass sch) throws SocketException, FileNotFoundException, IOException, ValidationException
    {
        //creating timezone
        TimeZoneRegistry registry = TimeZoneRegistryFactory.getInstance().createRegistry();
        TimeZone timezone = registry.getTimeZone("Asia/Jakarta");
        VTimeZone tz = timezone.getVTimeZone();
        
        //Start Date
        java.util.Calendar startDate = new GregorianCalendar();
        startDate.setTimeZone(timezone);
        startDate.set(java.util.Calendar.MONTH, sch.getDate().getMonthValue()-1);
        startDate.set(java.util.Calendar.DAY_OF_MONTH, sch.getDate().getDayOfMonth());
        startDate.set(java.util.Calendar.YEAR, sch.getDate().getYear());
        startDate.set(java.util.Calendar.HOUR_OF_DAY, sch.getTimeAwal().getHour());
        startDate.set(java.util.Calendar.MINUTE, sch.getTimeAwal().getMinute());
        
        //EndDate
        java.util.Calendar endDate = new GregorianCalendar();
        endDate.setTimeZone(timezone);
        endDate.set(java.util.Calendar.MONTH, sch.getDate().getMonthValue()-1);
        endDate.set(java.util.Calendar.DAY_OF_MONTH, sch.getDate().getDayOfMonth());
        endDate.set(java.util.Calendar.YEAR, sch.getDate().getYear());
        endDate.set(java.util.Calendar.HOUR_OF_DAY, sch.getTimeAkhir().getHour());
        endDate.set(java.util.Calendar.MINUTE, sch.getTimeAkhir().getMinute());
       
        
        //creating an event
        String eventName = sch.getSubject();
        String location2 = sch.getLocation();
        String desc = "Mengawas Ujian "+sch.getDosen();
        DateTime start = new DateTime(startDate.getTime());
        DateTime end = new DateTime(endDate.getTime());
        VEvent mengawas = new VEvent(start,end,eventName);
        mengawas.getProperties().add(new Location(location2));
        mengawas.getProperties().add(new Description());
        
        try {
            mengawas.getProperties().getProperty(Property.DESCRIPTION).setValue(desc);
        } catch (Exception e) {
        }
        
        //add timezone info
        mengawas.getProperties().add(tz.getTimeZoneId());
        
        //generate unique indentifier
        UidGenerator uidgenerator = new UidGenerator("uidGen");
        Uid uid = uidgenerator.generateUid();
        mengawas.getProperties().add(uid);
        
        
        //creating calendar
        net.fortuna.ical4j.model.Calendar calendar = new net.fortuna.ical4j.model.Calendar();
        calendar.getProperties().add(new ProdId("-//Ben Fortuna//iCal4j 1.0//EN"));
        calendar.getProperties().add(Version.VERSION_2_0);
        calendar.getProperties().add(CalScale.GREGORIAN);
        
        // Add the event and print
        calendar.getComponents().add(mengawas);
        System.out.println(calendar);
        
        //saving iCal
        String calFile = sch.getSubject();
        
        FileOutputStream fout = new FileOutputStream(path);
        
        CalendarOutputter outputter = new CalendarOutputter();
        outputter.setValidating(false);
        outputter.output(calendar, fout);    
    }   
}

\end{lstlisting}

\begin{lstlisting}[language=Java,basicstyle=\tiny,caption=FXMLDocumentController.java,label=lst:FXMLDocumentController]
package icalendarconverter;

import java.io.File;
import java.io.FileNotFoundException;
import java.io.IOException;
import java.net.SocketException;
import java.net.URL;
import java.util.ArrayList;
import java.util.ResourceBundle;
import javafx.beans.value.ChangeListener;
import javafx.beans.value.ObservableValue;
import javafx.collections.FXCollections;
import javafx.collections.ListChangeListener;
import javafx.collections.ObservableList;
import javafx.event.ActionEvent;
import javafx.fxml.FXML;
import javafx.fxml.Initializable;
import javafx.scene.control.Alert;
import javafx.scene.control.Alert.AlertType;
import javafx.scene.control.Label;
import javafx.scene.control.TextField;
import javafx.stage.FileChooser;
import javafx.scene.control.TableColumn;
import javafx.scene.control.TableView;
import javafx.scene.control.cell.PropertyValueFactory;
import javafx.util.Callback;
import javax.swing.JFileChooser;
import net.fortuna.ical4j.model.ValidationException;
/**
 *
 * @author Ariq
 */
public class FXMLDocumentController implements Initializable {
    private File selectedFile;
     
    @FXML
    private Label label;
    @FXML
    private TextField txtFile;
    @FXML
    private TextField filterTxt;
    @FXML
    private TableView<ScheduleClass> jadwalTable;
    
    ObservableList<ScheduleClass> jadwalList;
    ObservableList<ScheduleClass> filteredData = FXCollections.observableArrayList();
    
    @FXML
    private void handleButtonAction(ActionEvent event) {

          FileChooser fileChooser = new FileChooser();
          fileChooser.setTitle("Open Resource File");
          
          //Set extension filter
          FileChooser.ExtensionFilter extFilter = new FileChooser.ExtensionFilter("Excel files (*.xlsx)", "*.xlsx");
          FileChooser.ExtensionFilter extFilter2 = new FileChooser.ExtensionFilter("Excel files (*.xls)", "*.xls");
          fileChooser.getExtensionFilters().add(extFilter);
          fileChooser.getExtensionFilters().add(extFilter2);
          selectedFile = fileChooser.showOpenDialog(null);
         
          if (selectedFile != null)
          {
                 txtFile.setText(selectedFile.getAbsolutePath());
          }
          else 
          {
             
          }
    }
    
    @FXML
    private void handleConvertAction(ActionEvent event) throws FileNotFoundException, IOException
    {
       ExcelConverter con = new ExcelConverter(selectedFile);
        jadwalList = FXCollections.observableArrayList(con.Converter());
        
        
        jadwalTable.setItems(jadwalList);
        jadwalTable.getColumns().get(0).setCellValueFactory(new PropertyValueFactory("Date"));
        jadwalTable.getColumns().get(1).setCellValueFactory(new PropertyValueFactory<>("timeAwal"));
        jadwalTable.getColumns().get(2).setCellValueFactory(new PropertyValueFactory<>("timeAkhir"));
        jadwalTable.getColumns().get(3).setCellValueFactory(new PropertyValueFactory("Subject"));
        jadwalTable.getColumns().get(4).setCellValueFactory(new PropertyValueFactory("Dosen"));
        jadwalTable.getColumns().get(5).setCellValueFactory(new PropertyValueFactory("Location"));
        
        filteredData.addAll(jadwalList);
        
        jadwalList.addListener( new ListChangeListener<ScheduleClass>()
        {
            @Override
            public void onChanged(ListChangeListener.Change<? extends ScheduleClass> change)
            {
                updateFilteredData();
            }
        });
        
    }
    
    @Override
    public void initialize(URL url, ResourceBundle rb) {
      
        
    } 
    
    public void convertClicked() throws FileNotFoundException, 
            IOException, SocketException, ValidationException
    {
        ScheduleClass selected = jadwalTable.getSelectionModel().getSelectedItem();
        FileChooser fileChooser = new FileChooser();
        fileChooser.setTitle("Save iCal File");
        FileChooser.ExtensionFilter extFilter = new FileChooser.ExtensionFilter("iCalendar files (*.ics)", "*.ics");
        fileChooser.getExtensionFilters().add(extFilter);
        File save = fileChooser.showSaveDialog(null);
        
        int idx = jadwalTable.getSelectionModel().getSelectedIndex();
        String path;
        if(save != null)
        {
            path = save.getAbsolutePath();
            CalendarConverter cc = new CalendarConverter();
            cc.calConverter(path , selected);
        }
        else
        {
            System.out.println("Canceled !");
        }
    }
    @FXML
    private void filterConvertion()
    {
        jadwalTable.setItems(filteredData);
        filterTxt.textProperty().addListener(new ChangeListener<String>()
        {
            @Override
            public void changed(ObservableValue<? extends String> observable,
                    String oldValue, String newValue)
            {
                updateFilteredData();
            }
        });   
    }
    
    private void updateFilteredData()
    {
        filteredData.clear();;       
        for (ScheduleClass sc : jadwalList)
        {
            if (matchesFilter(sc))
            {
                filteredData.add(sc);
            }
        }       
        reapplyTableSortOrder();
    }
    
    private boolean matchesFilter(ScheduleClass sc)
    {
        String filterString = filterTxt.getText();
        
        if (filterString == null || filterString.isEmpty())
        {
            return true;
        }     
        String lowerCaseFilterString = filterString.toLowerCase();
        
        if (sc.getDosen().toLowerCase().indexOf(lowerCaseFilterString) != -1)
        {
            return true;
        }
        return false;
    }
    
    private void reapplyTableSortOrder()
    {
        ArrayList<TableColumn<ScheduleClass, ? >> sortOrder = new ArrayList<>(jadwalTable.getSortOrder());
        jadwalTable.getSortOrder().clear();
        jadwalTable.getSortOrder().addAll(sortOrder);
    }   
}

\end{lstlisting}