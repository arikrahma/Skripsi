%versi 2 (8-10-2016) 
\chapter{Pendahuluan}
\label{chap:intro}
   
\section{Latar Belakang}
\label{sec:latar_belakang}
\indent Jadwal mengawas ujian di FTIS(Fakultas Teknologi Infomasi dan Sains) merupakan hal yang rutin dibagikan filenya kepada dosen setiap semester. Jadwal mengawas tersebut dipublikasikan oleh TU(Tata Usaha). Format jadwal mengawas ujian bersifat umum, dalam arti jadwal tersebut menyimpan nama semua dosen yang mengawas, nama mata kuliah, dan tempat pelaksanaan ujian. Dosen diharuskan melihat satu persatu baris untuk mendapatkan informasi mengenai waktu, nama matakuliah, dan tanggal dosen tersebut mengawas. Walaupun jadwal mengawas tersebut telah disusun dalam file excel, namun tetap dirasa kurang efisien karena tidak tersusun berdasarkan dosen yang mengawas dan memungkinkan terjadi kesalahan dalam membaca jadwal oleh dosen. 
iCalendar merupakan format file \textit{calendar} pada komputer yang memudahkan penggunanya untuk mengirimkan undangan \textit{meeting} dan melakukan pekerjaan bersama pengguna lainnya, via email, atau file \textit{sharing} menggunakan ekstensi .ics . Format iCalendar sendiri telah didukung dan kompatibel dengan produk lainnya, seperti Google Calendar, Microsoft Outlook, Apple Calendar.
Tugas akhir ini dimaksudkan untuk memudahkan dosen untuk melihat jadwal mengawas ujian. Pengembangan perangkat lunak ini menggunakan tiga \textit{library} yaitu Apache POI, JavaFX, dan iCal4j.    

\section{Rumusan Masalah}
\label{sec:rumusan}
Berdasarkan penjelasan di latar belakang, maka dapat dipaparkan rumusan masalah sebagai berikut :
\begin{enumerate}
	\item Bagaimana perangkat lunak dapat membaca file excel jadwal mengawas ujian yang dibuat oleh TU ?
	\item Bagaimana perangkat lunak mengkonversi jadwal mengawas menjadi iCalendar ? 
	\item Bagaimana kompatibilitas dengan aplikasi lain ? 
\end{enumerate}


\section{Tujuan}
\label{sec:tujuan}
Tujuan dari skripsi ini dapat dipaparkan sebagai berikut:
\begin{enumerate}
	\item Merancang PL(Perangkat Lunak) yang mampu membaca file excel yang dipublikasikan oleh TU.
	\item Merancang PL(Perangkat Lunak) dapat mengkonversi jadwal mengawas menjadi iCalendar.
	\item Merancang PL(Perangkat Lunak) yang kompatibel dengan aplikasi lain.
\end{enumerate}

\section{Batasan Masalah}
\label{sec:batasan}
Batasan masalah dalam penelitian ini agar dapat fokus pada pengembangan perangkat lunak konversi jadwal mengawas ujian :
\begin{itemize}
	\item Diasumsikan format excel jadwal mengawas yang menjadi \textit{input} sudah fix.
	\item File excel jadwal mengawas berekstensi .xlsx.
	\item \textit{Sheet} yang dibaca dan dikonversi oleh PL adalah \textit{Sheet} pertama.
\end{itemize}

\section{Metodologi}
\label{sec:metlit}
Untuk menunjang penelitian maka diperlukan data untuk pengujian maupun pengetahuan teori yang akan diterapkan. Berikut adalah kegiatan yang akan dilakukan:
\begin{enumerate}
		\item Melakukan studi pustaka mengenai
			\begin{enumerate}
				\item Apache POI
				\item JavaFX
				\item iCal4j
			\end{enumerate}
		\item Melakukan analisis pada file excel jadwal mengawas ujian yang dikeluarkan oleh TU.
		\item Melakukan perancangan yang terdiri dari use case, diagram aktifitas, dan \textit{user interface}.
		\item Melakukan pengujian perangkat lunak dengan berbagai kemungkinan kasus.
		\item Menyimpulkan atas serangkaian pengembangan yang dilakukan
		\item Menulis dokumen skripsi
	\end{enumerate}

\section{Sistematika Pembahasan}
\label{sec:sispem}
\begin{enumerate}
	\item Bab 1 Pendahuluan\\
	Bab ini berisi tentang latar belakang, rumusan masalah, tujuan, batasan masalah, metodologi penelitian, dan sistematika pembahasan.
	\item Bab 2 Dasar Teori\\
	Bab ini berisi tentang teori dasar tentang File Excel, iCalendar, Apache POI, iCal4j, Java FX.
	\item Bab 3 Analisis\\
	Bab ini berisi tentang analisis kebutuhan dan fitur PL, diagram aktifitas PL, use case, diagram kelas.
	\item Bab 4 Perancangan\\
	Bab ini berisi tentang perancangan kelas dalam PL dan gambaran \textit{user interface}.
	\item Bab 5 Implementasi dan Pengujian\\
	Bab ini berisi tentang penerapan hasil rancangan pada bab sebelumnya serta pengujian perangkat lunak.
	\item Bab 6 Kesimpulan dan Saran\\
	Bab ini berisi tentang kesimpulan yang didapatkan dari hasil pengujian serta saran apabila ingin melanjutkan pengembangan ini.
\end{enumerate}